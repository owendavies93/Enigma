\documentclass[11pt, notitlepage]{report}

\usepackage[tmargin=2cm, lmargin=2cm, rmargin=2cm, bmargin=2cm]{geometry}
\usepackage[latin1]{inputenc}
\usepackage{chngpage}

\begin{document}

\setlength{\parindent}{0in}

\title{Enigma over IP (EoIP)}
\author{Written by Owen Davies}
\date{\today}

\maketitle
\renewcommand{\abstractname}{TL;DR}
\begin{abstract}
\noindent Send messages to other networked machines, using a simulated Enigma 
machine to encrypt the messages. It's like World War II but with computers! 
Credit for the stupid name goes to Daniel Hertz.
\end{abstract}
	
\section*{Usage}
EoIP is run from the commnad line, using the following command line argument
structure:
\begin{center}
\texttt{./enigma [*.rot] .pb [-n server|(client [hostname|IP]) port
 [-v]]}
\end{center}

\begin{itemize}
\item The \texttt{-n} flag tells the program that we want to run it with 
networking enabled, and either create a client or a server;
\item The next argument is either \texttt{server} or \texttt{[hostname|IP]}
, which tells the program whether we want to set up a server which will listen 
for a client or whether we want to set up a client who will look for a server on
 the machine called \texttt{hostname} or with IP address \texttt{IP};
\item The \texttt{port} argument tells the server which port to listen on, and 
tells the client which port to look on;
\item Finally, you can add the \texttt{-v} argument if you want to start the 
program in verbose mode (which shows diagnostic network messages and more 
detailed information about the encryption and decryption done by the Enigma 
machine).
\end{itemize}

Like in World War II, the server and the client machines must use the same 
Enigma machine rotor and plugboard configuration in order for the messages 
recieved to be decrypted correctly. Obivously a server \texttt{instance} must be
 running before a \texttt{client} instance.

\section*{Implementation Details}
The networking aspect of this program is implemented using the C/C++ standard 
socket networking library, \texttt{socket.h}. It uses the \texttt{sockaddr\_in} 
structure to configure the server, and then calls the \texttt{listen} function 
to wait for a connection to be made by a client.\\

Once the server is listening for connections, the client can then be run. The 
user must give the client the correct hostname or IP of the server, and the 
port which the server is listening for connections on. Once it has configured 
the server address settings, it uses the \texttt{accept} function from the 
\texttt{socket.h} library to attempt a connection to the server.\\

Once the server and client are connected they can alternate sending messages 
to eachother. The message system is implemented in this alternate fashion 
as this is how messages would have been sent back in the war (definitely 
NOT because it's quite a lot easier to code!). For example, the following 
output for a server instance (in verbose mode) of EoIP:\\

\texttt{./enigma rotors/II.rot rotors/III.rot plugboards/null.pb -n server 12345 -v\\
        Setting up socket...\\
        Configuring server address settings...\\
        Waiting for connection...\\
        Accepting connection...\\
        Connected to 127.0.0.1\\
        Reading encrypted message:\\
        LSHCV \\
        Unencrypting...\\
        127.0.0.1> HELLO \\
        Write message:\\
        > HELLO\\
        Encrypting message as:\\
        VPCHY \\
        Sending message to client...\\
        Message sent.}\\

Is mirrored by the following output from a client instance of enigma (also in 
verbose mode):\\

\texttt{./engima rotors/II.rot rotors/III.rot plugboards/null.pb -n client localhost 12345 -v
        Setting up socket...\\
        Configuring server address settings...\\
        Attemping to connect...\\
        Connected to 127.0.0.1\\
        Write message:\\
        > HELLO\\
        Encrypting message as:\\
        LSHCV \\
        Sending message to client...\\
        Message sent.\\
        Reading encrypted message:\\
        VPCHY \\
        Unencrypting...\\
        127.0.0.1> HELLO}\\

This exmaple shows how the server and the client are run with the same enimga 
configuration settings, and are run on the same port. Here I use the loop back 
IP to talk between two instances of engima on the same machine.
\end{document}
